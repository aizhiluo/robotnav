\subsection*{Overview}

The working environment of a robot is usually represented as a map. To find a path for the robot to traverse the environment, we need to extract information from the map and use proper data structures to represent the workspace. A variety of approaches have been proposed by researchers to find a path from the workspace which can satisfy specified requirements.

\subsubsection*{1. Planning Based on Discrete Search}

The workspace can be represented by smaller, connected areas so that discrete search algorithms can be used for path planning. As shown in the following figure, one needs to choose a proper method to decompose the workspace first and then use a graph to represent the pairwise relations between neighbour areas. With the graph, algorithms like A$\ast$ can be performed to find a sequence of areas that connects the starting and finishing points.

 Currently there are two methods provided to decompose the workspace\-: square grid and quadtree.

\subsubsection*{2. Sample-\/\-Based Motion Planning}

\subsection*{Modules}


\begin{DoxyItemize}
\item Square grid
\item Quadtree
\item \hyperlink{graph}{Graph}
\end{DoxyItemize}

\subsection*{Known Issues}


\begin{DoxyItemize}
\item A$\ast$ search algorithm currently only works with nodes that have attribute \char`\"{}location\-\_\-\char`\"{}. This attribute is used to calculate heuristic cost. A more general method may need to be implemented in the future. 
\end{DoxyItemize}